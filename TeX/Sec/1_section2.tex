\section{DRUGO POGLAVJE}
\subsection{Pisanje in referenciranje enačb}
Tako lahko napišeš enačbo $1+1=2$ v besedilu. Tako pa med vrsticami
$$1+1=2$$
ali z številčenjem enačb
\begin{equation}\repeatable{eq:1}
  {1+1=2}
\end{equation}
Pripravljen je ukaz za referenciranje, ki v AdobeAcrobat omogoči prikaz enačbe kadar z miško lebdiš nad $^?$.
To izgleda tako \tooltipref{eq:1} za običajen način sklicenavja pa piši \eqref{eq:1}.
Ukaz \verb|\tooltipref| deluje samo za okolje \verb|equation|. Problem ostalih okolij so simboli \& in \textbackslash\textbackslash .


\begin{figure}[h!]
  \centering
  \includegraphics[width = 5cm]{Img/UL_FGG-logoVER-CMYK_barv.pdf}
  \engcaption{Primer lepe slikice}{Example of a pretty figure}
\end{figure}


\begin{table}[h!]
  \centering
  \begin{tabular}{c | c}
    a & b\\ 
    \hline
    1 & 2\\
    2 & 3
  \end{tabular}
  \caption{Test}
\end{table}
